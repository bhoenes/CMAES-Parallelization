\documentclass[twocolumn, balance]{article}
\usepackage{amssymb}
\usepackage{amsmath}
\usepackage{amsfonts}
\usepackage{algorithm}
\usepackage{algorithmic}

\begin{document}

\title{CMAES}

\author{Brian Hoenes\\
    Colorado School of Mines\\
	Department of Mathematics and Computer Science\\
	1700 Illinois Street\\
    Golden, Colorado, USA\\
	bhoenes@mines.edu}

\maketitle
\begin{abstract}
This paper provides a sample of a \LaTeX\ document which conforms to
the formatting guidelines for ACM SIG Proceedings.
It complements the document \textit{Author's Guide to Preparing
ACM SIG Proceedings Using \LaTeX$2_\epsilon$\ and Bib\TeX}. This
source file has been written with the intention of being
compiled under \LaTeX$2_\epsilon$\ and BibTeX.

The developers have tried to include every imaginable sort
of ``bells and whistles", such as a subtitle, footnotes on
title, subtitle and authors, as well as in the text, and
every optional component (e.g. Acknowledgments, Additional
Authors, Appendices), not to mention examples of
equations, theorems, tables and figures.

To make best use of this sample document, run it through \LaTeX\
and BibTeX, and compare this source code with the printed
output produced by the dvi file.
\end{abstract}

\section{Introduction}

Localization of dipole

Inverse problems

\subsection{CMAES} 

\begin{algorithm}[h!]
\caption{Basic pCMAES}
\label{pCMAES}
\begin{algorithmic}
\REQUIRE{$\textit{obs} \in \mathbb{R}^O, \textit{Evaluate}(\mathbb{R}^D, \mathbb{R}^O) \Rightarrow \mathbb{R}, \lambda \in \mathbb{Z}_+$}
\FORALL{Processors $p\in P$}
	\STATE initialize Random Seed on $p$
	\STATE initialize mean $m$, covariance $C$, step size $\sigma$, path(s) \textit{path}
	\STATE $\mu \Leftarrow \frac{\lambda}{2}$
	\WHILE{NotDone}
		\FOR{$i \Leftarrow 1 \textrm{to} \frac{\lambda}{\| P \|}$}
			\STATE $q_i \Leftarrow N(m,C)$
			\STATE $v_i \Leftarrow \textit{Evaluate}(q_i, \textit{observations}$
		\ENDFOR
		\STATE $\textit{Sort}(v,q)$
		\STATE $m \Leftarrow \textit{mean}(q_1:q_\mu)$
		\STATE Update \textit{path} using $m$
		\STATE Update $\sigma$ using \textit{path}
		\STATE Update $C$ using \text{path}, $\sigma$
		\STATE Determine NotDone
	\ENDWHILE
	\STATE Reduce Best Solution
\ENDFOR
\end{algorithmic}
\end{algorithm}

\section{The body of The Paper}
Typically, the body of a paper is organized
into a hierarchical structure, with numbered or unnumbered


\end{document}
